%%%%%%%%%%%%%%%%%%%%%%%%%%%%%%%%%%%%%%%%%
% Beamer Presentation
% LaTeX Template
% Version 1.0 (10/11/12)
%
% This template has been downloaded from:
% http://www.LaTeXTemplates.com
%
% License:
% CC BY-NC-SA 3.0 (http://creativecommons.org/licenses/by-nc-sa/3.0/)
%
%%%%%%%%%%%%%%%%%%%%%%%%%%%%%%%%%%%%%%%%%

%----------------------------------------------------------------------------------------
%	PACKAGES AND THEMES
%----------------------------------------------------------------------------------------

\documentclass{beamer}

\mode<presentation> {

% The Beamer class comes with a number of default slide themes
% which change the colors and layouts of slides. Below this is a list
% of all the themes, uncomment each in turn to see what they look like.

%\usetheme{default}
%\usetheme{AnnArbor}
%\usetheme{Antibes}
%\usetheme{Bergen}
%\usetheme{Berkeley}
%\usetheme{Berlin}
%\usetheme{Boadilla}
%\usetheme{CambridgeUS}
%\usetheme{Copenhagen}
%\usetheme{Darmstadt}
%\usetheme{Dresden}
%\usetheme{Frankfurt}
%\usetheme{Goettingen}
%\usetheme{Hannover}
%\usetheme{Ilmenau}
%\usetheme{JuanLesPins}
%\usetheme{Luebeck}
\usetheme{Madrid}
%\usetheme{Malmoe}
%\usetheme{Marburg}
%\usetheme{Montpellier}
%\usetheme{PaloAlto}
%\usetheme{Pittsburgh}
%\usetheme{Rochester}
%\usetheme{Singapore}
%\usetheme{Szeged}
%\usetheme{Warsaw}

% As well as themes, the Beamer class has a number of color themes
% for any slide theme. Uncomment each of these in turn to see how it
% changes the colors of your current slide theme.

%\usecolortheme{albatross}
%\usecolortheme{beaver}
%\usecolortheme{beetle}
%\usecolortheme{crane}
%\usecolortheme{dolphin}
%\usecolortheme{dove}
%\usecolortheme{fly}
%\usecolortheme{lily}
%\usecolortheme{orchid}
%\usecolortheme{rose}
%\usecolortheme{seagull}
%\usecolortheme{seahorse}
%\usecolortheme{whale}
%\usecolortheme{wolverine}

%\setbeamertemplate{footline} % To remove the footer line in all slides uncomment this line
%\setbeamertemplate{footline}[page number] % To replace the footer line in all slides with a simple slide count uncomment this line

%\setbeamertemplate{navigation symbols}{} % To remove the navigation symbols from the bottom of all slides uncomment this line
}

\usepackage{graphicx} % Allows including images
\usepackage{booktabs} % Allows the use of \toprule, \midrule and \bottomrule in tables

%----------------------------------------------------------------------------------------
%	TITLE PAGE
%----------------------------------------------------------------------------------------

\title[]{Problem Formulation of LMS Filter} % The short title appears at the bottom of every slide, the full title is only on the title page

\author[K Samhitha , G Anusha] % (optional, for multiple authors)
{K Samhitha\inst{1} \and G Anusha\inst{2}}
\institute[IITH] % (optional)
{
  \inst{1}%
  EE16BTECH11019\\
  \and
  \inst{2}%
  EE16BTECH11011\\
  }

\date{\today} % Date, can be changed to a custom date

\begin{document}

\begin{frame}
\titlepage % Print the title page as the first slide
\end{frame}



%----------------------------------------------------------------------------------------
%	PRESENTATION SLIDES
%----------------------------------------------------------------------------------------

%------------------------------------------------

\begin{frame}
\frametitle{What is Least Mean Squares Algorithm ?}
Least mean squares (LMS) algorithms are a class of adaptive filter used to mimic a desired filter by finding the filter coefficients that relate to producing the least mean square of the error signal (difference between the desired and the actual signal). \\
\\
It is a stochastic gradient descent method in that the filter is only adapted based on the error at the current time.



\end{frame}

\begin{frame}
\frametitle{Problem Formulation}
The signal-noise waveform d(n) contains a human voice along with an instrument sound in the background.
This sound is captured in noise waveform X(n).
The goal is to suppress X(n) in 
d_n.\\
\begin{center}
\begin{equation*}
     d(n) = e(n) + y(n)
\end{equation*}
Where e(n) is the desired signal\\
\end{center}




\end{frame}




%------------------------------------------------

\begin{frame}
\frametitle{Cont..}
We want an estimate
of W(n) from X(n). This can be done by
considering
\begin{center}
\begin{equation*}
     y(n) = W^{T}(n)X(n)
\end{equation*}
\end{center}
 Where
 \begin{center}
\begin{equation*}
     X(n) = {\begin{pmatrix}

X(n)\\
X(n-1)\\
X(n-2)\\
..\\
..\\
X(n-M+1)\\
\end{pmatrix}}_{MX1}
\end{equation*}
\end{center}
\end{frame}


%------------------------------------------------

\begin{frame}{Cont..}
 \begin{center}
\begin{equation*}
     W(n) = {\begin{pmatrix}

w_1(n)\\
w_2(n)\\
w_3(n)\\
..\\
..\\
w_{n-M+1}(n)\\
\end{pmatrix}}_{MX1}
\end{equation*}
\end{center}
And estimating W(n).\\
The human voice can be characterized as
\begin{equation*}
    e(n) = d(n)-W^{T}(n)X(n)
\end{equation*}
\end{frame}

%------------------------------------------------

\begin{frame}
\frametitle{Cont..}
The goal is to find W(n) that will allow W^{T}(n)X(n)\\
to mimic the instrument sound in d(n). This is
possible if e(n) is minimum. This problem can be
expressed as
\begin{center}
\begin{equation*}
     \min_{W(n)} e^{2}(n)
\end{equation*}
   
\end{center}
\end{frame}



%----------------------------------------------------------------------------------------







\begin{frame}{And last}
\begin{center}
    Thank You
\end{center}
\end{frame}
\end{document} 
