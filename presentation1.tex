%%%%%%%%%%%%%%%%%%%%%%%%%%%%%%%%%%%%%%%%%
% Beamer Presentation
% LaTeX Template
% Version 1.0 (10/11/12)
%
% This template has been downloaded from:
% http://www.LaTeXTemplates.com
%
% License:
% CC BY-NC-SA 3.0 (http://creativecommons.org/licenses/by-nc-sa/3.0/)
%
%%%%%%%%%%%%%%%%%%%%%%%%%%%%%%%%%%%%%%%%%

%----------------------------------------------------------------------------------------
%	PACKAGES AND THEMES
%----------------------------------------------------------------------------------------

\documentclass{beamer}

\mode<presentation> {

% The Beamer class comes with a number of default slide themes
% which change the colors and layouts of slides. Below this is a list
% of all the themes, uncomment each in turn to see what they look like.

%\usetheme{default}
%\usetheme{AnnArbor}
%\usetheme{Antibes}
%\usetheme{Bergen}
%\usetheme{Berkeley}
%\usetheme{Berlin}
%\usetheme{Boadilla}
%\usetheme{CambridgeUS}
%\usetheme{Copenhagen}
%\usetheme{Darmstadt}
%\usetheme{Dresden}
%\usetheme{Frankfurt}
%\usetheme{Goettingen}
%\usetheme{Hannover}
%\usetheme{Ilmenau}
%\usetheme{JuanLesPins}
%\usetheme{Luebeck}
\usetheme{Madrid}
%\usetheme{Malmoe}
%\usetheme{Marburg}
%\usetheme{Montpellier}
%\usetheme{PaloAlto}
%\usetheme{Pittsburgh}
%\usetheme{Rochester}
%\usetheme{Singapore}
%\usetheme{Szeged}
%\usetheme{Warsaw}

% As well as themes, the Beamer class has a number of color themes
% for any slide theme. Uncomment each of these in turn to see how it
% changes the colors of your current slide theme.

%\usecolortheme{albatross}
%\usecolortheme{beaver}
%\usecolortheme{beetle}
%\usecolortheme{crane}
%\usecolortheme{dolphin}
%\usecolortheme{dove}
%\usecolortheme{fly}
%\usecolortheme{lily}
%\usecolortheme{orchid}
%\usecolortheme{rose}
%\usecolortheme{seagull}
%\usecolortheme{seahorse}
%\usecolortheme{whale}
%\usecolortheme{wolverine}

%\setbeamertemplate{footline} % To remove the footer line in all slides uncomment this line
%\setbeamertemplate{footline}[page number] % To replace the footer line in all slides with a simple slide count uncomment this line

%\setbeamertemplate{navigation symbols}{} % To remove the navigation symbols from the bottom of all slides uncomment this line
}

\usepackage{graphicx} % Allows including images
\usepackage{booktabs} % Allows the use of \toprule, \midrule and \bottomrule in tables

%----------------------------------------------------------------------------------------
%	TITLE PAGE
%----------------------------------------------------------------------------------------

\title[]{Problem 4.1} % The short title appears at the bottom of every slide, the full title is only on the title page

\author[K Samhitha , G Anusha] % (optional, for multiple authors)
{K Samhitha\inst{1} \and G Anusha\inst{2}}
\institute[IITH] % (optional)
{
  \inst{1}%
  EE16BTECH11019\\
  \and
  \inst{2}%
  EE16BTECH11011\\
  }

\date{\today} % Date, can be changed to a custom date

\begin{document}

\begin{frame}
\titlepage % Print the title page as the first slide
\end{frame}



%----------------------------------------------------------------------------------------
%	PRESENTATION SLIDES
%----------------------------------------------------------------------------------------

%------------------------------------------------

\begin{frame}
\frametitle{What is Semidefinite Programming?}
In semidefinite programming,one minimizes a linear function subject to the constraint that an affine combination of symmetric matrices is positive semidefinite.\\
Such a constraint is non linear and non smooth, but convex, so semidefinite programs are convex optimization problems.\\
They have a particular structure that makes their solution computationally tractable by interior-point methods.\\



\end{frame}

\begin{frame}
\frametitle{Question 4.1}
\begin{equation*}
    \min_{x} f(\textbf{x}) = x_{11} + x_{12}
\end{equation*}
With constraints\\

\begin{center}
$g_1(\textbf{x}) = x_{11} + x_{22} = 1$\\
$g_2(\textbf{x}) = \textbf{X} \geq 0$\\
$$
\textbf{X} = \begin{pmatrix}
x_{11}&x_{12}\\
x_{12}&x_{22}\\
\end{pmatrix}
$$
\end{center}

\end{frame}




%------------------------------------------------

\begin{frame}
\frametitle{Solution}
\textbf{cvxopt} solver is used to find the solution.Reformulate the given problem as,

\begin{center}
\begin{equation*}
    \min_{x} \begin{pmatrix}
1&1&0\\

\end{pmatrix}\begin{pmatrix}
x_{11}\\
x_{12}\\
x_{22}\\
\end{pmatrix}\\
s.t
\begin{pmatrix}
1&0&1\\

\end{pmatrix}
\begin{pmatrix}
x_{11}\\
x_{12}\\
x_{22}\\
\end{pmatrix}
= 1
\end{equation*}
\end{center}
\begin{equation*}
   x_{11}\begin{pmatrix}
-1&0\\
0&0\\
\end{pmatrix}
+ x_{12}\begin{pmatrix}
0&-1\\
-1&0\\
\end{pmatrix}
+x_{22}\begin{pmatrix}
0&0\\
0&-1\\
\end{pmatrix}
\leq \begin{pmatrix}
0&0\\
0&0\\
\end{pmatrix}
\end{equation*}
\end{frame}


%------------------------------------------------

\begin{frame}{General Format}
\begin{center}
Minimize Cx\\
Subject to Ax = b\\
and Gx \leq h\\
\end{center}

\end{frame}

%------------------------------------------------

\begin{frame}
\frametitle{Code}

from cvxopt import matrix\\
from cvxopt import solvers\\

c = matrix([1.,1.,0.])\\
G = [ matrix([[-1., 0., 0., 0.],[ 0., -1., -1., 0.],[0.,  0.,  0., -1.]]) ]\\

Aval = matrix([1.,0.,1.],(1,3))\\
bval = matrix([1.])\\

h = [ matrix([[0., 0.], [0., 0.]]) ]\\
sol = solvers.sdp(c, Gs=G, hs=h,A=Aval, b=bval)\\
print(sol['x'])  \\
print(sol['x'][0]+sol['x'][1]) \\
print('found at' ,sol['x'][0] ,'and' ,sol['x'][1]) \\


 

\end{frame}
\begin{frame}{Solution}
\begin{center}
    \includegraphics[scale=0.45]{4_1.png}
\end{center}
\end{frame}


%----------------------------------------------------------------------------------------







\begin{frame}{And last}
\begin{center}
    Thank You
\end{center}
\end{frame}
\end{document} 